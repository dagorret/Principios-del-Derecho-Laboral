\documentclass[a4paper]{article}

\usepackage[spanish]{babel}
\usepackage[utf8]{inputenc}
\usepackage{graphicx}
\usepackage[colorinlistoftodos]{todonotes}

\title{Reflexiones sobre los Principios del Derecho Laboral}

\author{Carlos M. Dagorret}

\date{\today}

\begin{document}
\maketitle

\begin{abstract}
Resumen bibliográfico escueto sobre los principios del Derecho Laboral, destacando los aspectos generales de ellos. 
Se enumera y se describen tratando de darle una mirada desde su integralidad y transversalidad.
El punto de vista adoptado como hipótesis social es la de Standing, que refiere a que el derecho laboral, en lo específico, debe ser parte de una integralidad, y que el trabajador en el sistema capitalista, de consumo y división de trabajo pasa a ser ciudadano solo si está formalmente ocupado.
Este trabajo no argumenta tal afirmación. Pero trata de describir los principios a analizar bajo el dinamismo que plantea el autor mencionado.
La línea seleccionada es la reflexiva, pre hipótesis, o exploratoria de conceptos.
\end{abstract}

\section{Aspectos socio-económicos }
\label{sec:introduccion}

En el proceso de globalización económica, las políticas de ajuste estructural y en general la modificación del modelo de desarrollo en el cual se encuentra Latinoamérica plantea diversos problemas al ámbito laboral, en mi opinión e hipótesis. 

En el campo del derecho laboral como en muchos otros campos del derecho, es frecuente enfrentarse a la preocupación sobre si las regulaciones legislativas continúan siendo válidas en cada etapa histórica, o si por el contrario deberían dichas normas ser reconsideradas o reformadas. \cite{Altimir}

En ese sentido interpreto que la concepción de los principios generales del derecho del trabajo, deriva del modelo de desarrollo económico-social y sobre todo de una concepción especifica del estado, de su papel en la sociedad, y sobre todo de su participación en la regulación de las relaciones laborales por referencia al asunto que se estudia. 

Es mi pensamiento que dichos cambios económicos sociales son explicados por los grandes movimientos globales en los sistemas de producción y reproducción del sistema capitalista: la globalización económica, la apertura comercial y la incipiente vuelta al proteccionismo global tienen o tendrán efectos directos sobre el mundo del trabajo y con ello sobre las relaciones laborales. 

En este marco los principios generales del derecho laboral son los más expuesto a la reputación\cite{RAEReputarse} por su carácter axiomático, integrador y transversal de esta rama del derecho específico  y las demás normativas sobre la economía en general (derecho comercial, impositiva, regímenes de promomoción,  normal financieras, etc.)

Por otra parte, y tal y como es lógico o de sentido común, los efectos, alteraciones o simplemente modificaciones en la realidad laboral de cada país, tienden a ser acompañados de propuestas concretas de modificación de la ley laboral e incluso de sentencias que orientan a iniciar nuevas interpretaciones de la legislación existente. 

Creo que hay que tener en cuenta que las regulaciones en materia de los principios del derecho laboral, ya fueron afectadas en el marco de la globalización económica, que se manifiesta en propuestas o reformas legislativas concretas. 

El análisis de los efectos de la globalización en el trabajo está fuera del alcance de este trabajo, pero de la bibliografía que se aporta se cree que los hechos, tal cual es el pensamiento de Standing, marcan a las sociedades. \cite{Standing2010}


\section{Los principios del Derecho del trabajo}
\label{sec:principios}

\subsection{Características y Funciones}
Más que características, se trata de un verdadero conjunto de “normas generales” que subyacen en todos el ordenamiento laboral y que cuentan en muchos casos también con un origen derivado de las normas de rango constitucional. 

En el derecho laboral el Estado se involucra en relaciones privadas protegiendo a la parte más débil de esa relación, precisamente en atención a tal tutela, la ley ordinaria recoge y establece una serie de manifestaciones concretas elevadas a la categoría de principios generales por la doctrina, que se encuentran entrelazadas entre sí por una genérica función tutelar del trabajador.\cite{Terrasa2017}

Hoy pareciera apropiado señalar que esa tutela que se reconoce a los principios generales del derecho laboral, obedece también al apuntalamiento de un modelo de desarrollo que propugna por relaciones laborales definidas y controladas en contenido y ejecución por el Estado. 

Estos principios generales, se encuentran en la base de todo el derecho laboral argentino e incluso el latinoamericano, y son también espacios en los que coinciden la mayoría de las legislaciones latinoamericanas.\cite{Terrasa2017}

Entre los principales pueden citarse el principio protector y sus reglas, el principio de la irrenunciabilidad y el principio de la continuidad de la relación; junto a ellos, coexisten también el principio de la condición más beneficiosa, el de igualdad de trato, de la gratuidad, de razonabilidad y el principio a favor de la duda, entre otros. 

Estos principios, al igual que los generales del derecho, poseen tres funciones netamente diferenciadas:\cite{Orsini2010}:
\begin{enumerate}
\item \textbf{Función Informativa}, ya que sirven de base e informan al legislador para la creación de nuevas normas y mantener la coherencia en todos los plexos legales.
\item \textbf{Función Interpretativa}, porque deben ayudar al intérprete de la norma para una correcta aplicación, en especial a los jueces, que son los encargados de aplicarla
\item \textbf{función normativa},  porque en ausencia de la norma el principio adquiere significativa importancia supliendo a la norma.
\end{enumerate}


\subsection{El principio protector}
Un reconocimiento prácticamente unánime establece que la ley laboral tiene como un verdadero principio general la protección del trabajador. 

Es un principio muy importante, es el fundamento de las leyes laborales, y justifica por si solo la intervención estatal en la emisión de las normas, en la vigilancia de su cumplimiento efectivo, y en la aplicación especifica. 

Este, denominado por la doctrina como el principio protector del derecho laboral, contiene  reglas especificas que impulsa la interpretación más favorable al trabajador(\textbf{"In dubio pro operario"}), la regla de la condición más beneficiosa, o la regla de la norma más favorable. 

La importancia del principio protector es tal, que en realidad se le llega a ubicar en todo el contenido de la ley laboral; trasciende por así decirlo, un ámbito restringido. 

Además busca armonizar las relaciones entre el capital y el trabajo y se funda en principios que tienden al mejoramiento de las condiciones de vida de los trabajadores. 

\subsection{El principio de la irrenunciabilidad}
Este principio define como la imposibilidad jurídica de privarse voluntariamente de una o más ventajas consentidas por el derecho laboral por parte de cualquier trabajador. 

Este principio se relaciona, en mi opinión, con la naturaleza de orden público que se reconoce a las normas laborales. 

Este principio establece una regla: las renuncias que los trabajadores formulen a las disposiciones que les favorezcan son nulas.

En ese sentido no solo se declara en la ley que los derechos son irrenunciables, sino que toda renuncia, disminución o tergiversación de esos derechos se declaran “nulos por ipso jure” – nulos por virtud del derecho.

\subsection{El principio de continuidad de la relación.}

El principio de la continuidad puede definirse como aquel según el cual también en beneficio del trabajador, se establecen una serie de reglas que definen a las relaciones laborales como dotadas de una extrema vitalidad y dureza y que realizan o evidencian la tendencia del derecho del trabajo por atribuirle la más larga duración a la relación laboral desde todos los puntos de vista y en todos los aspectos. 

El principio orienta así el anhelo por relaciones labores regidas por un régimen de verdadera estabilidad en el empleo. 

Se introduce con ello una restricción importante al mercado laboral, la que resulta admisible solo cuando previamente se ha admitido la intervención del Estado en las relaciones privadas, con el mismo se pretende brindar alguna estabilidad a los trabajadores, obligados a la venta de su fuerza de trabajo como medio de subsistencia. 

De esta forma la finalización de las relaciones laborales se legisla bajo la presunción de un hecho de alta relevancia para la ley laboral, siendo objeto como tal de una fuerte regulación dotada de algunas características particulares. 

\subsection{Principio de la condición más beneficiosa}

El principio de la condición más beneficiosa se puede interpretar como derecho adquirido a favor del trabajar o una prerrogativa jurídica antes las situaciones de la relación de trabajo. 

Esto se refiere a las normas originadas en normas contractuales individuales, expresas o tácitas, pero solo de alcance individual. 

El principio es simple : ante situaciones legales planteadas  por dos o más normas que beneficia a un trabajador, se debe considerar para la situación que se debe decidir la más beneficio le otorga a ese trabajador. 

\subsection{Principio de igualdad de trato}

El principio de igualdad de trato se concreta en que, de parte del empleador, el trabajador reciba un tratamiento igual, para iguales en iguales circunstancias. 

Esa igualdad equivale, en mi opinión, a la prohibición de un trato diferente arbitrario. 

Por lógica, de este principio se deduce el respectivo derecho a no ser tratado arbitrariamente con desigualdad. 

\subsection{El principio de favor de la duda}

En su origen, el principio in dubio pro operario implicó invertir el principio vigente en el derecho privado según el cual los casos dudosos deben resolverse a favor del deudor. 

Este consiste en otorgar un amparo (interpretarse como cobertura) a la parte más débil en el contrato de trabajo;  que precisamente a consecuencia de su debilidad se encuentra en la mayoría de los casos en la situación de acreedor.\cite{Luco2014}

De este modo, y contrastandoló en forma general este es un principio de excepción y el in dubio pro operario ha pasado a ser un principio más específico versus el principio generalizado del favor de la duda. 

Este último, el de favor ante la duda, recoge una tendencia de tomar en cuenta específicamente la posición de debilidad estructural en el mercado. \cite{Luco2014}

Para la aplicación del principio de favor de la duda, implica que existe dos extremos contradictorios y que debe ser superables mediante un análisis de mérito de la situación del trabajador cuando está en la zona gris de los extremos. 

\subsection{Principio de la primacía de la Realidad}

Lo que hace referencia es que será nulo todo contrato por el cual las partes hayan procedido con simulación o fraude a la ley laboral, sea aparentando normas contractuales no laborales o interposición de terceras personas.

Es por ello que en caso de discordancia entre los documentos o acuerdos escritos y lo que sucedió en la realidad, se prefiere lo último. 

Pues lo que persigue el principio es la verdad real y no la verdad formal. 

\subsection{Principio de la Buena Fe}

El Derecho del Trabajo pretende que la relación laboral se desenvuelva con conductas propias de un buen empleador y de un buen trabajador, con fidelidad, lealtad y veracidad.

Este principio debe ser tenido en cuenta para la aplicación de todos los derechos y obligaciones que ambas del contrato de trabajo. 

Se puede interpretar como un principio que rige el modo de actuar, un estilo de conducta, una forma de proceder ante las relaciones laborales que se da en el ámbito del trabajo por las partes. 

\subsection{Principio de la Razonabilidad}

Por este principio entendemos la afirmación esencial de que el ser humano en sus relaciones laborales, procede y debe proceder conforme a la razón. 

Una obviedad implícita que debería ser transversal en todos los ámbitos de la sociedad. 

En el campo del Derecho Laboral la aplicación de este principio actuaría a mi criterio en dos sentidos: 

Uno, serviría para medir la verosimilitud de determinada explicación o solución. 

Y dos, también sirve como cauce, como límite, como freno de ciertas facultades cuya amplitud puede prestarse para arbitrariedades. 

Creo que este principio es importante en las situaciones que el empleador varia las condiciones de trabajo y el régimen disciplinario;  ya que como toda sanción en el sistema legal, tiene un principio de proporcionalidad. 

\subsection{Principio de gratuidad}

Es el principio por el cual se le garantiza al trabajador el derecho de defensa a través de la gratuidad de los procedimientos, de modo que dicho derecho no resulte comprometido por el costo económico que podría significar para el trabajador, cuando dicho costo pueda ser una limitación para su ejercicio. 

Este principio apunta fundamentalmente a que el trabajador tenga acceso a la justicia de manera libre y sin entorpecimiento. 

También se expresa en la gratuidad de los procedimientos administrativos (Ministerio de Trabajo), y en la posibilidad de obtener patrocinio letrado gratuito, poder emitir notificaciones, poder peticionar y librar acciones de oficios judiciales, etc.

\subsection{Principio de Equidad}

Establece, a mi criterio, cuando de la aplicación de la ley el resultado  que se pueda obtener sea notoriamente injusto, de cumplimiento oneroso, nocivo o perjudicial para el trabajador.

Este principio es un principio corrector de las normas que se aplica al ámbito laboral y que reside en la amerituación del juez su aplicabilidad.

Si bien la evaluación es el magistrado quien la hace, este no debe apartarse del espíritu de la ley y de la razón de la existencia de dicha norma.

\subsection{Principio de Justicia Social}

Es un principio orientador de que las sentencia judiciales deben buscar el otorgamiento de lo que le corresponde al trabajador observando el bien común dada la naturaleza de órden público de la legislación laboral.

Quizás, es un principio muy amplio y difuso. Pero es una directriz que se deberá considerar para suplir zonas grises en los plexos normativos.

\subsection{Principio de Progresividad}

El principio de progresividad es relativamente novedoso en el derecho del trabajo argentino. 

Establece que ningún cambio se puede realizar en el marco del contrato de trabajo que implique una disminución o pérdida de un derecho, y en su caso, los cambios o modificaciones son sólo admisibles si son más beneficiosas para el trabajador. 

Es más, el Estado debe propiciar las mejoras o reformas que contribuyan a respetar los derechos enunciados por el sistema legal  en cuanto a su calidad y extensión, y debería adicionar los medios o mecanismos para que gradualmente, los derechos no sólo se apliquen, sino que además incorporen nuevos elementos en beneficio del trabajador. 

El principio de progresividad también es denominado principio de irregresividad. \cite{Ramirez2010}

\newpage
\section{Referencias y Notas}
\label{sec:latex}
\subsection{Notas sobre el formato}

Para este trabajo se utilizó la planilla "Quantum Hall effect report", recomendada para "Homework Assignment" de la Universidad de Copenhague.

El código fuente del documento se encuentra en:

https://es.overleaf.com/read/sfxgjqfrxhbt


\begin{thebibliography}{9}
\bibitem{Altimir}
Altimir, Oscar, y Luis Alberto Beccaria. \emph{El mercado de trabajo bajo el nuevo régimen económico en Argentina.} (1999).Pág. 19-27.
\bibitem{RAEReputarse}
Reputar: Juzgar o hacer concepto del estado o calidad de alguien o algo. Según la RAE, versión en línea, visto el 7 de octubre 2018.
\bibitem{Standing2010}Standing, Guy. \emph{Work after globalization: Building occupational citizenship}. Edward Elgar Publishing, 2010
\bibitem{Terrasa2017}Terrasa L. \emph{Los Derechos Fundamentales en las Relaciones Laborales Privadas de Argentina}.Revista Latinoamericana de Derecho Social Núm. 25, julio-diciembre de 2017, pp. 157-180
\bibitem{Orsini2010}
Orisini, J. Ignacio.\emph{Los principios del derecho del trabajo}. Apuntes de cátedra de Derecho  Social. Facultad  de 
Ciencias Jurídicas y Sociales. UNLP. 2010.
 \bibitem{Luco2014}
 Luco, Munita. \emph{El principio protector y la regla del in dubio pro operario como criterio de interpretación de la normal laboral}. Revista Chilena de Derecho del Trabajo y de la Seguridad Social. Vol. 5, Nro 10, pag. 85-94. 2014
 \bibitem{Ramirez2010}
   Ramírez Bosco, Luis, “El principio de progresividad y de irregresividad”, en Ley de Contrato de Trabajo Comentada, editorial La Ley, volumen I página 332.
\end{thebibliography}
\end{document}